%%% DOCUMENT TYPE %%%%%%%%%%%%%%%%%%%%%%%%%%%%%%%%%%%%%%%%%%%%%%%%%%%%%%%%%%%%%%



%%% SETUP %%%%%%%%%%%%%%%%%%%%%%%%%%%%%%%%%%%%%%%%%%%%%%%%%%%%%%%%%%%%%%%%%%%%%%

%%% DOCUMENT TYPE %%%%%%%%%%%%%%%%%%%%%%%%%%%%%%%%%%%%%%%%%%%%%%%%%%%%%%%%%%%%%%

\documentclass[9pt, a4paper, twocolumn, landscape]{extarticle}

%%% PACKAGES %%%%%%%%%%%%%%%%%%%%%%%%%%%%%%%%%%%%%%%%%%%%%%%%%%%%%%%%%%%%%%%%%%%

% Encoding

\usepackage[utf8]{inputenc}
\usepackage[T1]{fontenc}

% Geometry

\usepackage{geometry} % edit margins of paper
\usepackage{setspace} % edit line spacing
\usepackage{fancyhdr} % header, footer
\usepackage{titlesec} % edit format of titles

% Visual

\usepackage[dvipsnames]{xcolor} % colors
\usepackage{tikz} % graphics
\usepackage[framemethod=tikz]{mdframed} % frames, better theorems

% Math

\usepackage{amsmath} % math tools
\usepackage{amssymb} % math symbols
\usepackage{amsthm} % thereoms
\usepackage{mathtools} % math tools

% Referencing

\usepackage{nameref}
\usepackage{hyperref}
\usepackage{cleveref}

% Useful

\usepackage[shortlabels]{enumitem} % enumerations

% Other

\usepackage{lastpage} % get number of last page
\usepackage{physics}
\usepackage{bbm}

%%% MARGINS %%%%%%%%%%%%%%%%%%%%%%%%%%%%%%%%%%%%%%%%%%%%%%%%%%%%%%%%%%%%%%%%%%%%

\geometry{a4paper, landscape, left=10mm, right=10mm, top=10mm, bottom=10mm, includehead}

%%% COLORS %%%%%%%%%%%%%%%%%%%%%%%%%%%%%%%%%%%%%%%%%%%%%%%%%%%%%%%%%%%%%%%%%%%%%

%%% TITLES %%%%%%%%%%%%%%%%%%%%%%%%%%%%%%%%%%%%%%%%%%%%%%%%%%%%%%%%%%%%%%%%%%%%%

\colorlet{color-section}                {Blue}
\colorlet{color-subsection}             {RoyalPurple}
\colorlet{color-paragraph}             {MidnightBlue}
\colorlet{color-subsubsection}	{CadetBlue}
%%% MATH BOXES %%%%%%%%%%%%%%%%%%%%%%%%%%%%%%%%%%%%%%%%%%%%%%%%%%%%%%%%%%%%%%%%%

\colorlet{color-definition}              {Blue!20}%{SpringGreen!20}
\colorlet{color-theorem}                {Brown!25}%{Apricot!13}
\colorlet{color-proposition}            {ProcessBlue!13}% {Apricot!13}
\colorlet{color-corollary}              {Salmon!12}%{Apricot!13}
\colorlet{color-lemma}                  {Brown!7}%{Apricot!13}
\colorlet{color-remark}                 {Gray!4}
\colorlet{color-example}                {Lavender!7}
% \colorlet{color-proof}                  {FILL COLOR HERE}


%%% CAPTIONS %%%%%%%%%%%%%%%%%%%%%%%%%%%%%%%%%%%%%%%%%%%%%%%%%%%%%%%%%%%%%%%%%%%

\input{setup/captions.tex}

%%% SHORTCUTS %%%%%%%%%%%%%%%%%%%%%%%%%%%%%%%%%%%%%%%%%%%%%%%%%%%%%%%%%%%%%%%%%%

\input{setup/shortcuts.tex}

%%% FORMATTING %%%%%%%%%%%%%%%%%%%%%%%%%%%%%%%%%%%%%%%%%%%%%%%%%%%%%%%%%%%%%%%%%

\input{setup/formatting.tex}

%%% LANGUAGE %%%%%%%%%%%%%%%%%%%%%%%%%%%%%%%%%%%%%%%%%%%%%%%%%%%%%%%%%%%%%%%%%%%

\input{setup/languages/EN.tex}
\usepackage[arrow, matrix, curve]{xy}
\usepackage{wrapfig}
\usepackage{bm}
\usepackage{multicol}
\usepackage{xcolor}
\usepackage{mathrsfs} 
\renewcommand\vec{\boldsymbol}

%%% DOCUMENT %%%%%%%%%%%%%%%%%%%%%%%%%%%%%%%%%%%%%%%%%%%%%%%%%%%%%%%%%%%%%%%%%%%

\begin{document}

\section{Stuff dump}

\paragraph{Cambell Baker Hausdorf} Für alle $t \in \mathbb{R}$ 
$$ 
\exp(t A) \exp(t B)= \exp \left(t A+t B+\frac{t^2}{2}[A, B]
+\frac{t^3}{12} [A,[A, B]] +\frac{t^3}{12}[B,[B, A]]+\mathscr{O}\left(t^4\right)\right)
$$

\paragraph{Creation-/Annihilation Operators}in second quantisation:
$$\left[\hat{a}_i^*, \hat{a}_j^*\right]=0, \quad\left[\hat{a}_i, \hat{a}_j\right]=0, \quad\left[\hat{a}_i, \hat{a}_j^*\right]=\delta_{i, j} \hat{\mathrm{id}}$$


\paragraph{Bernoulli Trial} Probability of $k$ successes in a bernoulli experiment $B(n,p)$:
$$P(k)=\left(\begin{array}{l}
  n \\
  k
  \end{array}\right) p^k q^{n-k} $$

\paragraph{Bayes' rule}  $$
\operatorname{Pr}[A \mid B \wedge C]=\frac{\operatorname{Pr}[B \mid A \wedge C]}{\operatorname{Pr}[B \mid C]} \operatorname{Pr}[A \mid C]
$$

\paragraph{Maxwell equations} we have \\
$ \nabla E = \frac{\rho}{\epsilon_0}  \quad  $  \ (Gauss law)   \\
$ \nabla B = 0 $ \\
$ \nabla \times E = \ - \frac{\partial B}{\partial t} \quad  $ \  (Faraday's law of induction) \\
$ \nabla \times B = \frac{1}{c^2} \frac{ \partial E}{\partial t} + \mu_0 J + \epsilon_0 \frac{\partial P} {\partial t} \quad $ \ (Ampere's law)


\paragraph{Bloch sphere}$\hat{n}(\theta, \phi)$ on Bloch sphere with $\theta \in (0, \pi)$, $\phi \in (0,2 \pi)$. 
For $- \hat{n}$ we have $(\theta, \phi) \to (\pi - \theta , \phi + \pi) $.\\
\includegraphics[scale=0.2]{fig/BlochSph.png}

\paragraph{Monte Carlo Methods} Class of algorithms that rely on random sampling to obtain numerical results

\paragraph{Delta distribution}
\begin{itemize}
  \item $\int d x e^{i k \cdot x}=(2 \pi) \delta(k)$
  \item $\int_{-\infty}^{+\infty} d x \delta(g(x))=\sum_i \frac{1}{\left|g^{\prime}\left(x_i\right)\right|} $
\end{itemize}

%%% Math stuff dump 

\paragraph{Fourier transform } For $\varphi \in \mathscr{S}\left(\mathbb{R}^n\right)$:\\
(i) $\left(\partial_j \varphi\right)^{\wedge}(k)=i k_j \hat{\varphi}(k)$\\
(ii) $\partial_j \hat{\varphi}(k)=\frac{\partial}{\partial k_j} \hat{\varphi}(k)=\left(-i x_j \varphi\right)^{\wedge}(k)$\\
(iii) $\left(\partial_j \varphi\right)^{\vee}(k)=-i k_j \check{\varphi}(k)$\\
(iv) $\partial_j \check{\varphi}(k)=\left(i x_j \varphi\right)^{\vee}(k)$\\
(v) $\mathcal{F}_x\left[e^{-a x^2}\right](k)=\sqrt{\frac{\pi}{a}} e^{k^2 a} \quad $ Normalsation 1; osc factor 1


%%%%%%%%%%%%%%%%%%% Quantum mechanics I & II
\section{Quantum Mechanics I \& II}

\paragraph{Propagator} characterised through \\
(i) $U(t, t)=\mathbf{I}$.\\
(ii) additivity/ unitar: $U(t, s) U(s, r)=U(t, r)$.\\
(iii) The operator $U(t, s)$ satisfies the differential equation
$$
i \hbar \partial_t U(t, s)=H U(t, s)
$$
For $H$ time independent: $U(t, s)=\exp \left(-i H \frac{(t-s)}{\hbar}\right)$\\
General: $U(t, s)=\mathcal{T} \exp \left[-\frac{i}{\hbar} \int_s^t d t^{\prime} H\left(t^{\prime}\right)\right]$
For $[H(t),H(s)]=0 \ \forall t,s$ we can omit the time order operator $\mathcal{T}$

\paragraph{Heisenberg picture} Time dependecy is shifted from states to operators:\\
$$\Psi_H= \Psi(t_0) = U\left(t_0, t\right) \Psi_S(t) \quad \quad A_H=U\left(t_0, t\right) A_S U\left(t, t_0\right)$$
The equation of motion in the Heisenberg picture: 
$$i \hbar \frac{d}{d t} A_H(t) = \left[A_H, H_H\right]+i \hbar \partial_t A_H$$
For $\partial_t H=0$ we have $H_H=H_S$ and $A_H(t)=e^{i H\left(t-t_0\right) / \hbar} A e^{-i H\left(t-t_0\right) / \hbar}$\\
\emph{The Heisenberg picture shows the similaritiy to classical mechanics where we have $\frac{d A}{d t}=\{A, H\}+\partial_t A$.
Replacing the Poisson braket with commutators and imposing the canonical commutator relations gives rise to qunatisation. }

\paragraph{Interaction (Dirac) picture} For $H=H_0+H^{\prime}(t)$. Idea is to shift (trivial) time dependence
of states originating from $H_0$ on to operators: $\Psi_D(t)=U_D\left(t, t_0\right) \Psi_D\left(t_0\right) $ with 
$U_D\left(t, t_0\right)=U_0\left(t_0, t\right) U\left(t, t_0\right)$ where $U$ is the Propagator for $H=H_0 +H'$. We have 
$$i \hbar \partial_t U_D = =H_D^{\prime} U_D$$


%%%%%%%%%%%%%%%%%%%%%%%%%%%%%%%%%%%%%%%%%%%%%%%%%%%%%%%%%%%%%%%%%%%%%%%%%%%%%%% QIT
\section{Quantum Information Theory}
\paragraph{Quantum probability}
$Pr[|\Lambda] = \Trace{\Lambda \rho} $\\
probabity density: $\quad \rho \in \operatorname{Lin}(\mathcal{H}), \quad \rho \geq 0, \quad \Trace[p] =1$\\
effect / measuremnt: $\Lambda \in \operatorname{Lin}(\mathcal{H}), \quad \Lambda \geq 0, \quad \Lambda \leq \mathbb{1}$\\

positivity of operators: $S \geq 0$ if $\langle v|S| v\rangle \geq 0 \text { for all } v \in \mathcal{H}$

\paragraph{POVM: postive operator valued measure} set of effects $\{ \Lambda (x)\}^n_{x=1}$ such that
$\Lambda(x) \in \operatorname{Lin}(\mathcal{H}) : \Lambda(x) \geq 0 \ \forall x, \ \sum_x \Lambda(x) =$ $1$

\paragraph{Trace (abstract)} $\Trace{\ketbra*{\Phi}{\Psi}} := \braket*{\Psi}{\Phi}$, then extend linearly. Then we have further\\
$\Trace[ABC]= \Trace[CAB]$ and for basis transformations we habe $\Tr[U \rho U^* ]= \Tr[\rho]$\\

\subsection{Composite Systems} 
$\mathcal{H}_{AB} = \mathcal{H}_A \otimes \mathcal{H}_B$\\
$\ket{+}_A\otimes \ket{+}_B = \ket*{++}_{AB} = \frac{1}{2} (\ket*{00}_{AB}+ \ket*{01}_{AB}+ \ket*{10}_{AB}+\ket*{11}_{AB})$\\
product state: $\ket{\Psi}\otimes \ket*{\phi} = \ket*{\Psi}_{AB} \quad$ entagled state: $\ket*{\Psi}_{AB}$ such that it cannot be written as productstate.\\
Partial trace $\Tr_{AB}[M_{AB}] = \Tr_A[\Tr_B[M_{AB}]]$

\paragraph{Technical stuff}
\begin{itemize}
  \item Pauli Operators: $\sigma_1=\left(\begin{array}{cc}
    0 & 1 \\
    1 & 0
    \end{array}\right), \quad \sigma_2=\left(\begin{array}{cc}
    0 & -\mathrm{i} \\
    \mathrm{i} & 0
    \end{array}\right), \quad \sigma_3=\left(\begin{array}{cc}
    1 & 0 \\
    0 & -1
    \end{array}\right)$
  \item $\left[\sigma_i, \sigma_j\right]=2 i \varepsilon_{i j k} \sigma_k$
  \item porbability density is pure state iff: $\Tr[p^2]=1$
  \item positivity of operators: $S \geq 0$ if $\langle v|S| v\rangle \geq 0 \text { for all } v \in \mathcal{H}. \quad $ $\longrightarrow \quad$ $S$ is hermitian


\end{itemize}


%%%%%%%%%%%%%%%%%%%%%%%%%%%%%%%%%%%%%%%%%%%%%%%%%%%%%%%%%%%%%%%%%%%%%%%%%%%%%%% QFT
\section{Quantum Field Theory I}

\paragraph{Basis transformation} $\{ \ket*{i} \to \ket*{\lambda} \}$  for orthonormal Basis:\\ 
$\ket*{\lambda } = \sum_i \ket*{i} \ketbra*{i}{\lambda} \quad   \implies  \quad $ if $\hat{a}_i^\dagger \ket*{0} = \ket*{i}$ then 
$\hat{a}_\lambda^\dagger \ket*{0} = \sum_i \ketbra*{i}{\lambda} \hat{a}_i^\dagger \ket{0} = \ket*{\lambda} $\\

Like this any Hamitonian of the form $H=T+U+V$ (e.g.) \\ 
$H=\sum_{i=1}^N \frac{\mathbf{p}_i^2}{2 m}-\sum_{i=1}^N \frac{Z e^2}{\left|\mathbf{x}_i\right|}+\sum_{i>j} \frac{e^2}{\left|\mathbf{x}_i-\mathbf{x}_j\right|}$
can be written as:\\
$H=\sum_{i, j} a_i^{\dagger}\langle i|T| j\rangle a_j+\sum_{i, j} a_i^{\dagger}\langle i|U| j\rangle a_j+\frac{1}{2} \sum_{i j k m}\langle i, j|V| k, m\rangle a_i^{\dagger} a_j^{\dagger} a_k a_m$

\paragraph{Klein Gordan equation} for real scalar fields. $\varphi (x) = \bar{\varphi}(\bar{x})$ implies that the equations of motion are the same:
$$ (- \partial^2 + m^2)\phi(x)$$ with $\hbar \text{ and } c = 1$
General solution given by $\varphi(x)=\int \widetilde{d k}\left[a(\mathbf{k}) e^{i k x}+a^*(\mathbf{k}) e^{-i k x}\right]$\\
with $ \widetilde{d k} \equiv \frac{d^3 k}{(2 \pi)^3 2 \omega}$ and $a(\mathbf{k}$ arbitrary function of $\mathbf{k}$. Only quatization and the canonical commutation relations unveil $a(\mathbf{k})$ as annhilation operator. 
We imposed that $\varphi(x)$ is real and introduced a Lorentz invariant differential for convience. $k x=\mathbf{k} \cdot \mathbf{x}-\omega t$ is the Lorentz four product. 

\subsection{Lorentzinvariance}


\paragraph{Lorentstransformations} $\left(\Lambda^{-1}\right)_{\ \nu}^\rho=\Lambda_\nu^{\ \rho}$

\paragraph{Lorentz stuff dump}
\begin{itemize}
  \item invaraince integration measure: $d^4 \bar{x}=|\operatorname{det} \Lambda| d^4 x=d^4 x$
  \item inverse Lorentz transformation: $\left(\Lambda^{-1}\right)_{\ \nu}^\rho=\Lambda_\nu^{\ \rho}$
  \item $K^\mu K_\mu=\left(\frac{\omega}{c}\right)^2-k_x^2-k_y^2-k_z^2=\left(\frac{\omega_o}{c}\right)^2=\left(\frac{m_o c}{\hbar}\right)^2$ with $K^\mu = \left( \frac{\omega}{c} , \vec{k} \right)$
\end{itemize}







\end{document}
