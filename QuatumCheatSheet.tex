%%% DOCUMENT TYPE %%%%%%%%%%%%%%%%%%%%%%%%%%%%%%%%%%%%%%%%%%%%%%%%%%%%%%%%%%%%%%



%%% SETUP %%%%%%%%%%%%%%%%%%%%%%%%%%%%%%%%%%%%%%%%%%%%%%%%%%%%%%%%%%%%%%%%%%%%%%

%%% DOCUMENT TYPE %%%%%%%%%%%%%%%%%%%%%%%%%%%%%%%%%%%%%%%%%%%%%%%%%%%%%%%%%%%%%%

\documentclass[9pt, a4paper, twocolumn, landscape]{extarticle}

%%% PACKAGES %%%%%%%%%%%%%%%%%%%%%%%%%%%%%%%%%%%%%%%%%%%%%%%%%%%%%%%%%%%%%%%%%%%

% Encoding

\usepackage[utf8]{inputenc}
\usepackage[T1]{fontenc}

% Geometry

\usepackage{geometry} % edit margins of paper
\usepackage{setspace} % edit line spacing
\usepackage{fancyhdr} % header, footer
\usepackage{titlesec} % edit format of titles

% Visual

\usepackage[dvipsnames]{xcolor} % colors
\usepackage{tikz} % graphics
\usepackage[framemethod=tikz]{mdframed} % frames, better theorems

% Math

\usepackage{amsmath} % math tools
\usepackage{amssymb} % math symbols
\usepackage{amsthm} % thereoms
\usepackage{mathtools} % math tools

% Referencing

\usepackage{nameref}
\usepackage{hyperref}
\usepackage{cleveref}

% Useful

\usepackage[shortlabels]{enumitem} % enumerations

% Other

\usepackage{lastpage} % get number of last page
\usepackage{physics}
\usepackage{bbm}

%%% MARGINS %%%%%%%%%%%%%%%%%%%%%%%%%%%%%%%%%%%%%%%%%%%%%%%%%%%%%%%%%%%%%%%%%%%%

\geometry{a4paper, landscape, left=10mm, right=10mm, top=10mm, bottom=10mm, includehead}

%%% COLORS %%%%%%%%%%%%%%%%%%%%%%%%%%%%%%%%%%%%%%%%%%%%%%%%%%%%%%%%%%%%%%%%%%%%%

%%% TITLES %%%%%%%%%%%%%%%%%%%%%%%%%%%%%%%%%%%%%%%%%%%%%%%%%%%%%%%%%%%%%%%%%%%%%

\colorlet{color-section}                {Blue}
\colorlet{color-subsection}             {RoyalPurple}
\colorlet{color-paragraph}             {MidnightBlue}
\colorlet{color-subsubsection}	{CadetBlue}
%%% MATH BOXES %%%%%%%%%%%%%%%%%%%%%%%%%%%%%%%%%%%%%%%%%%%%%%%%%%%%%%%%%%%%%%%%%

\colorlet{color-definition}              {Blue!20}%{SpringGreen!20}
\colorlet{color-theorem}                {Brown!25}%{Apricot!13}
\colorlet{color-proposition}            {ProcessBlue!13}% {Apricot!13}
\colorlet{color-corollary}              {Salmon!12}%{Apricot!13}
\colorlet{color-lemma}                  {Brown!7}%{Apricot!13}
\colorlet{color-remark}                 {Gray!4}
\colorlet{color-example}                {Lavender!7}
% \colorlet{color-proof}                  {FILL COLOR HERE}


%%% CAPTIONS %%%%%%%%%%%%%%%%%%%%%%%%%%%%%%%%%%%%%%%%%%%%%%%%%%%%%%%%%%%%%%%%%%%

\input{setup/captions.tex}

%%% SHORTCUTS %%%%%%%%%%%%%%%%%%%%%%%%%%%%%%%%%%%%%%%%%%%%%%%%%%%%%%%%%%%%%%%%%%

\input{setup/shortcuts.tex}

%%% FORMATTING %%%%%%%%%%%%%%%%%%%%%%%%%%%%%%%%%%%%%%%%%%%%%%%%%%%%%%%%%%%%%%%%%

\input{setup/formatting.tex}

%%% LANGUAGE %%%%%%%%%%%%%%%%%%%%%%%%%%%%%%%%%%%%%%%%%%%%%%%%%%%%%%%%%%%%%%%%%%%

\input{setup/languages/EN.tex}
\usepackage[arrow, matrix, curve]{xy}
\usepackage{wrapfig}
\usepackage{bm}
\usepackage{multicol}
\usepackage{xcolor}
\usepackage{mathrsfs} 
\renewcommand\vec{\boldsymbol}

%%% DOCUMENT %%%%%%%%%%%%%%%%%%%%%%%%%%%%%%%%%%%%%%%%%%%%%%%%%%%%%%%%%%%%%%%%%%%

\begin{document}

\section{Stuff dump}

\paragraph{Cambell Baker Hausdorf} Für alle $t \in \mathbb{R}$ 
$$ 
\exp(t A) \exp(t B)= \exp \left(t A+t B+\frac{t^2}{2}[A, B]
+\frac{t^3}{12} [A,[A, B]] +\frac{t^3}{12}[B,[B, A]]+\mathscr{O}\left(t^4\right)\right)
$$
 


\paragraph{Bernoulli Trial} Probability of $k$ successes in a bernoulli experiment $B(n,p)$:
$$P(k)=\left(\begin{array}{l}
  n \\
  k
  \end{array}\right) p^k q^{n-k} $$

\paragraph{Bayes' rule}  $$
\operatorname{Pr}[A \mid B \wedge C]=\frac{\operatorname{Pr}[B \mid A \wedge C]}{\operatorname{Pr}[B \mid C]} \operatorname{Pr}[A \mid C]
$$

\paragraph{Maxwell equations} we have \\
$ \nabla E = \frac{\rho}{\epsilon_0}  \quad  $  \ (Gauss law)   \\
$ \nabla B = 0 $ \\
$ \nabla \times E = \ - \frac{\partial B}{\partial t} \quad  $ \  (Faraday's law of induction) \\
$ \nabla \times B = \frac{1}{c^2} \frac{ \partial E}{\partial t} + \mu_0 J + \epsilon_0 \frac{\partial P} {\partial t} \quad $ \ (Ampere's law)


\paragraph{Bloch sphere}$\hat{n}(\theta, \phi)$ on Bloch sphere with $\theta \in (0, \pi)$, $\phi \in (0,2 \pi)$. 
For $- \hat{n}$ we have $(\theta, \phi) \to (\pi - \theta , \phi + \pi) $.\\
\includegraphics[scale=0.2]{fig/BlochSph.png}

\paragraph{Monte Carlo Methods} Class of algorithms that rely on random sampling to obtain numerical results

%%%%%%%%%%%%%%%%%%%%%%%%%%%%%%%%%%%%%%%%%%%%%%%%%%%%%%%%%%%%%%%%%%%%%%%%%%%%%%% QIT
\section{Quantum Information Theory}
\paragraph{Quantum probability}
$Pr[|\Lambda] = \Trace{\Lambda \rho} $\\
probabiity density: $\quad \rho \in \operatorname{Lin}(\mathcal{H}), \quad \rho \geq 0, \quad \Trace[p] =1$\\
effect / measuremnt: $\Lambda \in \operatorname{Lin}(\mathcal{H}), \quad \Lambda \geq 0, \quad \Lambda \leq \mathbb{1}$

\paragraph{POVM: postive operator valued measure} set of effects $\{ \Lambda (x)\}^n_{x=1}$ such that
$\Lambda(x) \in \operatorname{Lin}(\mathcal{H}) : \Lambda(x) \geq 0 \ \forall x, \ \sum_x \Lambda(x) =$ $1$

\paragraph{Trace (abstract)} $\Trace{\ketbra*{\Phi}{\Psi}} := \braket*{\Psi}{\Phi}$, then extend linearly. Then we have further\\
$\Trace[ABC]= \Trace[CAB]$ and for basis transformations we habe $\Tr[U \rho U^* ]= \Tr[\rho]$\\

\subsection{Composite Systems} 
$\mathcal{H}_{AB} = \mathcal{H}_A \otimes \mathcal{H}_B$\\
$\ket{+}_A\otimes \ket{+}_B = \ket*{++}_{AB} = \frac{1}{2} \ket*{00}_{AB}+ \ket*{01}_{AB}+ \ket*{10}_{AB}+\ket*{11}_{AB}$\\
product state: $\ket{\Psi}\otimes \ket*{\phi} = \ket*{\Psi}_{AB} \quad$ entagled state: $\ket*{\Psi}_{AB}$ such that it cannot be written as productstate
Partial trace $\Tr_{AB}[M_{AB}] = \Tr_A[\Tr_B[M_{AB}]]$


%%%%%%%%%%%%%%%%%%%%%%%%%%%%%%%%%%%%%%%%%%%%%%%%%%%%%%%%%%%%%%%%%%%%%%%%%%%%%%% QFT
\section{Quantum Field Theory I}

\paragraph{Basis transformation} $\{ \ket*{i} \to \ket*{\lambda} \}$  for orthonormal Basis:\\ 
$\ket*{\lambda } = \sum_i \ket*{i} \ketbra*{i}{\lambda} \quad   \implies  \quad $ if $\hat{a}_i^\dagger \ket*{0} = \ket*{i}$ then 
$\hat{a}_\lambda^\dagger \ket*{0} = \sum_i \ketbra*{i}{\lambda} \hat{a}_i^\dagger \ket{0} = \ket*{\lambda} $\\

Like this any Hamitonian of the form $H=T+U+V$ (e.g.) \\ 
$H=\sum_{i=1}^N \frac{\mathbf{p}_i^2}{2 m}-\sum_{i=1}^N \frac{Z e^2}{\left|\mathbf{x}_i\right|}+\sum_{i>j} \frac{e^2}{\left|\mathbf{x}_i-\mathbf{x}_j\right|}$
can be written as:\\
$H=\sum_{i, j} a_i^{\dagger}\langle i|T| j\rangle a_j+\sum_{i, j} a_i^{\dagger}\langle i|U| j\rangle a_j+\frac{1}{2} \sum_{i j k m}\langle i, j|V| k, m\rangle a_i^{\dagger} a_j^{\dagger} a_k a_m$

\subsection{Lorentzinvariance}






\end{document}
